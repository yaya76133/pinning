%% ****** Start of file aiptemplate.tex ****** %
%%
%%   This file is part of the files in the distribution of AIP substyles for REVTeX4.
%%   Version 4.1 of 9 October 2009.
%
%
% This is a template for producing documents for use with 
% the REVTEX 4.1 document class and the AIP substyles.
% 
% Copy this file to another name and then work on that file.
% That way, you always have this original template file to use.

\documentclass[aip,jcp,preprint]{revtex4-1}
%\documentclass[aip,reprint]{revtex4-1}

\usepackage[dvipdfmx]{graphicx}
\usepackage{amsmath, amssymb}% Include figure files
%
\usepackage{color}
\usepackage{ulem}
\def\rem#1{\textcolor{red}{\xout{#1}}}
\def\add#1{\textcolor{red}{{\bf #1}}}
%
\def\glv{\gamma_\mathrm{LV}}
\def\gsl{\gamma_\mathrm{SL}}
\def\gsv{\gamma_\mathrm{SV}}
\def\gs0{\gamma_\mathrm{S0}}
\def\ptl{\partial}
\def\ie{\textit{i.e.}, }
%
\def\eqref#1{(\ref{#1})}
\def\angb#1{\left<#1\right>}
\def\bm#1{\mbox{\boldmath $#1$}}
%
\def\dint{\int\!\!\!\int}
\def\tint{\int\!\!\!\int\!\!\!\int}
%
%\draft % marks overfull lines with a black rule on the right

\begin{document}

% Use the \preprint command to place your local institutional report number 
% on the title page in preprint mode.
% Multiple \preprint commands are allowed.
%\preprint{}

\title{Quantification of chemical pinning induced at wetting inhomogeneous wall.
  }
% repeat the \author .. \affiliation  etc. as needed
% \email, \thanks, \homepage, \altaffiliation all apply to the current author.
% Explanatory text should go in the []'s, 
% actual e-mail address or url should go in the {}'s for \email and \homepage.
% Please use the appropriate macro for the type of information

% \affiliation command applies to all authors since the last \affiliation command. 
% The \affiliation command should follow the other information.
%
\author{H. Kusudo}
%\email{hiroki@gcom.mech.eng.osaka-u.ac.jp}
%\homepage{http://www-gcom.mech.eng.osaka-u.ac.jp/~yamaguchi/.}
\affiliation{Department of Mechanical Engineering,
Osaka University, 2-1 Yamadaoka, Suita 565-0871, Japan}
%
\author{Y. Yamaguchi}
%\email{yamaguchi@mech.eng.osaka-u.ac.jp}
\homepage{http://www-gcom.mech.eng.osaka-u.ac.jp/.}
\affiliation{Department of Mechanical Engineering,
Osaka University, 2-1 Yamadaoka, Suita 565-0871, Japan}
\affiliation{Water Frontier Science \& Technology Research Center (W-FST),
Research Institute for Science \& Technology,
Tokyo University of Science,
1-3 Kagurazaka, Shinjuku-ku, Tokyo, 162-8601, Japan}

\date{\today}

\begin{abstract}

abstract......

\end{abstract}

\pacs{}% insert suggested PACS numbers in braces on next line

\maketitle %\maketitle must follow title, authors, abstract and \pacs

% Body of paper goes here. Use proper sectioning commands. 
% References should be done using the \cite, \ref, and \label commands
%
\section{Introduction}
intro

\begin{equation}
\label{feq:Young}
\gsl-\gsv+\glv \text{cos}\theta=0
\end{equation}
%
\section{Simulation method}
In order to examine pinning phenomenon induced by inhomogeneous wettability of wall surface as shown in Fig.~\ref{fig:systems}(a),
we employed two types of equilibrium simulation systems 
with respect to flat solid surfaces with inhomogeneous wettability, as shown in Fig.~\ref{fig:systems}:(b) for the contact line pinning and (c) liquid pinning calculation.
MD simulation method is almost same as the past paper~\cite{Yamaguchi2019} .
In all calculation systems, we used mono-atomic fluid molecules(Argon) and mono-atomic wall molecules 
whose structure is face-centered crystal(FCC).
Hereafter, fluid and wall molecules are denoted by `f' and `w', respectively.

\begin{figure}
  \begin{center}
    \includegraphics[width=100mm]{./figs/fig1.pdf}
  \end{center} 
  \caption{\label{fig:systems}
(a)Schematic of pinning induced by heterogeneous wettability.
    Lennard-Jones(LJ) liquid simulation systems on flat surfaces 
with inhomogeneous wettabilty which induces pinning of (b)the contact line
and (c)liquid.
  }
\end{figure}

\subsection{Potential}
Generic particles interacting through a 12-6 Lennard-Jones potential were adopted as the fluid molecules 
for ease of physical understanding as in our previous study.
The 12-6 LJ potential given by 
\begin{equation}
  \label{eq:LJ}
  \Phi^\mathrm{LJ}(r_{ij})=
  4\epsilon_\mathrm{ff} \left[
    \left(\frac{\sigma_\mathrm{ij}}{r_{ij}}\right)^{12} 
    -
    \left(\frac{\sigma_\mathrm{ij}}{r_{ij}}\right)^{6} 
   \right]
\end{equation}
was used as fluid-fluid and fluid-wall interaction.
In this 12-6 LJ potential, $r_{ij}$ was the distance between the
molecules $i$ at position 
$\bm{r}_{i}$ and $j$ at $\bm{r}_{j}$, while 
$\epsilon_\mathrm{ij}$ and $\sigma_\mathrm{ij}$ 
denoted the LJ energy and length parameters, 
respectively. 
This LJ interaction was truncated at a cut-off 
distance of $r^\mathrm{c}_\mathrm{ij}=3.5\,\sigma_\mathrm{ij}$ 
and quadratic functions were added so that the potential 
and interaction force smoothly vanished at $r_{c}$
as same as in our previous study~\cite{Yamaguchi2019}.

As above, the structure of solid wall is FCC and
 its inter-atomic potential is expressed by the following harmonic potential given by
\begin{equation}
\Phi^\mathrm{harmonic}(r_{ij})=
\frac{k}{2}(r_{ij}-r_{0})^{2},
\end{equation}
where $k$ and $r_{0}$ denote the spring constant and equilibrium distance, respectively.
These parameters and the mass of the wall atom corresponding with the property of platinum.

In addition to these intermolecular potentials, we used a one-dimentional field given by
\begin{equation}
\label{eq:Potwall}
\Phi^\mathrm{P}(z'_{i})=
4\pi\rho_{\text{n}}\epsilon^{0}_{fw}\sigma_{fw}^{2}\left[
\frac{1}{5}\left( \frac{\sigma_{fw}}{z'_{i}}\right)^{10}
-\frac{1}{2}\left( \frac{\sigma_{fw}}{z'_{i}}\right)^{4}
\right]
\end{equation}
as a function of the distance $z'_{i}=|z_{i}-z_{\text{pw}}|$
between a fluid particle and a potential wall.
The potential and mass parameters are summarized in Table~\ref{tab:table1}.

\subsection{Simulation systems}
For both of the calculation systems, liquid bath was put between two solid walls parallel to yz-plane and
periodic boundary conditions were set in y-direction as shown in Fig.~\ref{fig:systems}(b)(c),
so that liquid was shaped quasi two dimension: the effect of line tension was neglected.
The position of the outermost layers of solid walls consisting FCC was fixed, 
and the temperature of the second outermost layers was controlled 
by using the standard Langevin thermostat in all directions at a control temperature $T_{\text{c}}$
with a Debye temperature $T_{\text{Debye}}$.
Solid walls have eight layers 
so that solid wall atoms of  the second outermost layer don't interact with liquid molecules.

The wettability of the solid walls is inhomogeneous, lower and upper layers are lyophilic and lyophobic, respectively.
Solid liquid interaction coefficients of lyophilic and lyophobic parts are $\eta_{\text{phil}}$ and $\eta_{\text{phob}}$, respectively.
For contact line pinning system, a mirror boundary condition was set at the top boundary in z-direction
and a potential wall with a potential field Eq.~\eqref{eq:Potwall} was fixed at various positions $z_{\text{pw}}$,
so that we made equillibrium systems in Fig.~\ref{fig:systems}(a).
For liquid pinning system, the potential wall was fixed at a position $z_{\text{pw}}=1.4$~nm,
and additionally we used a piston with a potential field Eq.~\eqref{eq:Potwall} to make liquid single-phase 
under the following equation of motion:
\begin{equation}
m_{\text{pis}}\frac{d^{2} z_{\text{pis}}}{dt^{2}}=F_{\text{pis}}^{\text{int}}-A_{\text{pis}}p_{\text{pis}}
\end{equation}
where $z_{\text{pis}}$, $A_{\text{pis}}$, $m_{\text{pis}}$ and $p_{\text{set}}$ 
is a position, area, weight and control pressure of piston,respectively.
In the liquid pinning system, the piston cannot control the liquid pressure exactly
because the piston contacts with the wall surface inducing the negligible effect of the solid liquid interfacial tension.
The internal force on the piston $F_{\text{pis}}^{\text{int}}$ is the total normal force exerted by fluid molecules.
%with a potential field Eq.~\eqref{eq:Potential} at $z_{\text{pw}}=z_{\text{pis}}$

The velocity Verlet method was applied for the integration of the Newtonian equation of motion 
with a time increment $\Delta t$ of 5 fs for all systems. 
Values of the simulation parameters are summarized in Table~\ref{tab:table1} with the corresponding non- dimensional ones, 
which are normalized by the corresponding standard values.  

\begin{table}[!t]
\caption{\label{tab:table1} 
Simulation parameters. Corresponding non-dimensional values
are shown below.
}
%\begin{tabular*}{100mm}{@{\extracolsep{\fill}}|c|c|c|} \hline\hline
\begin{ruledtabular}
\begin{tabular}{cccc}
property  & value & unit & non-dim. value
\\ \hline
$\sigma_\mathrm{ff}$ & 0.340 & nm & 1
\\
$\sigma_\mathrm{ww}$ 
\footnote{Used only for Lorentz mixing rule.}
& 0.350 & nm & 1.03 
\\
$\sigma_\mathrm{fw}$ & $(\sigma_\mathrm{ff} + \sigma_\mathrm{ww})/2$ 

\\
$\epsilon_\mathrm{ff}$ & $1.67 \times 10^{-21}$ & J & 1
\\
$\epsilon_\mathrm{ww}$ 
\footnote{Used only for Berthelot mixing rule.}
& $1.000\times 10^{-21}$ & J & 0.599
\\
$\epsilon^{0}_\mathrm{fw}$ & 
$ \sqrt{\epsilon_\mathrm{ff}\epsilon_\mathrm{ww}}$
\\
$\epsilon_\mathrm{fw}$ & 
$ \eta_{\text{phil}} \epsilon^{0}_\mathrm{fw}$ or $\eta_{\text{phob}} \epsilon^{0}_\mathrm{fw}$
\\
$\eta_{\text{phil}}$ &
0.5& - & -
\\
$\eta_{\text{phob}}$ &
0.001 - 0.5& - & -
\\
$m_\mathrm{f}$ & $6.63 \times 10^{-26}$ & kg & 1
\\
$m_\mathrm{w}$ & $32.4\times 10^{-26}$ & kg & 4.88
\\
$m_{\text{pis}}$ & $14.7\times 10^{-22}$ & kg & $2.22\times10^{4}$
\\
$r_0$ (FCC) & $0.277$ & nm & 0.815
\\
$k$ & 46.8  & N/m & 3.24$\times10^{3}$
\\
$T_\mathrm{c}$ & 85  & K & 0.703
\\
$N_\mathrm{f}$ (contact line pinning) & 5000  & - & -
\\
$N_\mathrm{f}$ (liquid pinning) & 10000 & - & -
\\
$p_\mathrm{set}$ &1.00$\times 10^{6}$ & nm & 2.35$\times10^{-2}$
\\
$A_\mathrm{pis}$ &43.6 & $\text{(nm)}^{2}$ & 3.77$\times10^{2}$
\end{tabular}
\end{ruledtabular}
%
\end{table}

%
\section{Results and discussion}
\label{sec:result}
\section{Young's equation considering pinning force}
The time-averaged density distribution of
contact line pinning system for $z_\text{pin}=4.0~\text{nm}$ is shown in Fig.~\ref{fig:dens}.
\begin{figure}
  \begin{center}
    \includegraphics[width=120mm]{./figs/fig2.pdf}
  \end{center} 
  \caption{\label{fig:dens}
Time-averaged density distribution of contact line pinning system for $z_\text{pin}=4.0~\text{nm}$
used as an example system for the analysis of vertical force balance exerted on a rectangular control volume, shown in magenta, set around the contact line.
  }
\end{figure}
We supposed a control volume involving the contact line and comprised by four faces:
right face set at the center of the system,
left face set in vacuum layer i.e. solid liquid transition area at liquid layer,
top and bottom faces both normal to the wall and set outside the liquid vapor transition layer.
Assuming the total force on argon in the control volume exerted by the wall atoms as $\zeta_{\text{pin}}$,
the momentum conservation law can be written by the force balance on the control volume~\cite{Yamaguchi2019}:
\begin{equation}
\zeta_\text{pin}=\int_{x_{\text{L}}}^{x_{\text{C}}}\tau_{zz}(x,z_\text{T})dx-
\int_{x_{\text{L}}}^{x_{\text{C}}}\tau_{zz}(x,z_\text{B})dx+
\int_{z_{\text{B}}}^{z_{\text{T}}}\tau_{xz}(x_{\text{C}},z)dz
\end{equation}
,where the third term on right hand must be zero because of symmetric property.
This equation can be deformed as below with isotropic stress term $\tau^{\text{bulk}}_{\text{L}}$ and $\tau^{\text{bulk}}_{\text{V}}$
which have the opposite sign of pressures $p_{\text{int}}$ and $p_{\text{ext}}$.
\begin{equation}
\label{eq:FB1}
\zeta_\text{pin}=\int_{x_{\text{L}}}^{x_{\text{C}}}\tau_{zz}(x,z_\text{T})-\tau^{\text{bulk}}_{\text{V}}(x,z_\text{T})dx-
\int_{x_{\text{L}}}^{x_{\text{C}}}\tau_{zz}(x,z_\text{B})-\tau^{\text{bulk}}_{\text{L}}(x,z_\text{T})dx-
(p_{\text{int}}-p_{\text{ext}})(x_{\text{C}}-x_{\text{L}})
\end{equation}
Applying Young-Laplace's equation and Bakker's equation to Eq.~\eqref{eq:FB1}~\cite{Yamaguchi2019}, 
anyone can get the revised Young's equation:
\begin{eqnarray}
\zeta_{\text{pin}}
&=&(\gsv|_{\eta_{\text{phob}}}-\gs0)-(\gsl|_{\eta_{\text{phil}}}-\gs0)-\frac{x_{\text{C}}-x_{\text{L}}}{R}\glv \\
&=&\gsv|_{\eta_{\text{phob}}}-\gsl|_{\eta_{\text{phil}}}-\glv \text{cos}\theta
\label{eq:rY}
\end{eqnarray}
,where $\theta$ is a contact angle and geometrical relation $\text{cos}\theta=(x_{\text{C}}-x_{\text{L}})/R$ was applied.
If a contact angle is calculated, we can calculate the pinning force $\zeta_{\text{pin}}$ from this Eq.~\eqref{eq:rY}.
%%i.e. the total force on Ar fluids around the discontinuous wetting line exerted by wall atoms.
We show the comparison of two forces,
the total tangential force on Ar particles exerted by wall atoms around the wetting discontinuous line i.e. pinning force
and the force calculated by revised Young's equation~\eqref{eq:rY}, in Figure.~\ref{fig:clpinning} 
The value of $\gsv|_{\eta_{\text{phob}}}$ and $\gsl|_{\eta_{\text{phil}}}$are the resultant of Thermodynamic Integration~\cite{Yamaguchi2019}
and $\glv=11.3\times 10^{-3}\text{N/m}$~\cite{Nishida14}.
In the Fig~\ref{fig:clpinning}, the pinning force and the force calculated by the rivised Young's equation are
slightly different because of heat fluctuation and the assumption that liquid vapor interfacial tension is constant not depending on the curvature of meniscus.
However, we confirmed the validity of the revised Young's Eq.~\eqref{eq:rY} qualitatively 
because Fig.~\ref{fig:clpinning} showed the similar response of two forces to the position of potential wall $z_{\text{pw}}$,and
the response showed the existence of saturation value of pinning force.
\begin{figure}
  \begin{center}
    \includegraphics[width=100mm]{./figs/fig4.pdf}
  \end{center} 
  \caption{\label{fig:clpinning}
    The comparison of pinning force and the force calculated by revised Young's equation for various positions of potential wall
with screenshots of the calculation system.
  }
\end{figure}

\section{Quantification of the pinning force on solid liquid interface}
The pinning force has a saturation value according to Fig.~\ref{fig:clpinning}.
When the position of the potential wall $z_{\text{pw}}$ get high and the wetting discontinuous line is covered by Ar liquids,
the pinning force must reach maximum value according to Fig.~\ref{fig:systems}(a): 
the bigger pinning force would be induced, the more Ar molecules around the wetting discontinuous line 
because the pinning force is the total force of the unbalance force on Ar molecule exerted by the two different wetting area. 
We can calculate the relation between saturation value of the pinning force and Thermodynamics 
with liquid pinning system in Fig.~\ref{fig:systems}(c) 
through a thought experiment as shown in Fig.~\ref{fig:SLpin}.
Assuming the rectangular control volume with all faces set at vacuum layers to include all Ar fluids
,we can calculate the pinning force by the force balance on Ar liquids:
\begin{figure}
  \begin{center}
    \includegraphics[width=100mm]{./figs/fig5.pdf}
  \end{center} 
  \caption{\label{fig:SLpin}
    Thought experiment of liquid pinning. Red and yellow solid line denote the initial positions of potential wall and piston,respectively.
The dotted lines denote the position after the quasi-statistical displacement.
  }
\end{figure}

\begin{equation}
\label{eq:SLF}
2l \zeta_{\text{pin}}=F_{\text{pw}}-F_{\text{pis}}^{\text{int}},
\end{equation}
where $l$ and $F_{\text{pw}}$ are system length in y-direction  and 
the total force on the potential wall exerted by Ar fluids,respectively.
In addition, assuming a virtual quasi-statistical displacement of the bottom wall and piston as 
$\delta z_{\text{pin}}$ and $\delta z_{\text{pis}}$, respectively, 
the minimum mechanical work $\delta W$ 
required for this change is associated with the change in the Gibss free energy G given by
\begin{equation}
\label{eq:MW}
\delta G\equiv \delta W =F_{\text{pin}}\delta z_{\text{pin}}-F_{\text{pis}}\delta z_{\text{pis}}.
\end{equation}
Here, this Gibs's free energy change $\delta G$ is given by 
the product of the interface area change and the interfacial free energy as
\begin{equation}
\label{eq:FE}
\delta G=\left(\gsl|_{\eta_{\text{phob}}}-\gs0 \right) \times \delta z_{\text{pis}} 2l
-\left(\gsl|_{\eta_{\text{phil}}}-\gs0 \right) \times \delta z_{\text{pin}} 2l.
\end{equation}
Let $F_{\text{pw}}$ be canceled by applying Eq.~\eqref{eq:SLF} into Eq.~\eqref{eq:MW},
and after deformation of Eq.~\eqref{eq:MW} with Eq.~\eqref{eq:FE}
,pinning force $\zeta_{\text{pin}}$ is given as
\begin{eqnarray}
\zeta_{\text{pin}}&=&
\frac{
\left(\gsl|_{\eta_{\text{phob}}}-\gs0\right)\delta z_{\text{pis}}
-
\left(\gsl|_{\eta_{\text{phil}}}-\gs0\right)\delta z_{\text{pw}}
}{\delta z_{\text{pw}}}
+\frac{\delta z_{\text{pis}}-\delta z_{\text{pw}}}
{2l\delta z_{\text{pw}}}
F_{\text{pis}} \nonumber \\
&=&
\frac{
W_{\text{SL}}|_{\eta_{\text{phil}}}\delta z_{\text{pw}}
-
W_{\text{SL}}|_{\eta_{\text{phob}}}\delta z_{\text{pis}}
}
{\delta z_{\text{pw}}}
+\frac{\delta z_{\text{pis}}-\delta z_{\text{pw}}}
{\delta z_{\text{pw}}}
\left(
\glv+\frac{F_{\text{pis}}}{2l}
\right)
\label{eq:pin0}
\end{eqnarray}
,where $W_{\text{SL}}\equiv\glv+\gs0-\gsv$ is the work of adhesion of solid-liquid interface.
The difference of the adsorption layer density distribution would cause the gap between $z_{\text{pis}}$ and $z_{\text{pw}}$
that is negligibly small when the channel width is enough big.%or the wetting of two area is almost same.
Therefore, assuming $z_{\text{pis}}\simeq z_{\text{pw}}$
,we can simply express the pinning force as the difference of the work of adhesion by deforming Eq.~\eqref{eq:pin0}. 
\begin{equation}
\zeta_{\text{pin}}=W_{\text{SL}}|_{\eta_{\text{phil}}}-W_{\text{SL}}|_{\eta_{\text{phob}}}
\label{eq:pin1}
\end{equation}
We compared the pinning force and 
the precalculated~\cite{Yamaguchi2019} work of adhesion at two different wetting areas calculated by using Dry-Surface method~\cite{Leroy2015},
one of Thermodynamic Integration,
to show the validity of this Eq.~\eqref{eq:pin1} at Fig.~\ref{fig:pinSL}.
\begin{figure}
  \begin{center}
    \includegraphics[width=120mm]{./figs/fig6.pdf}
  \end{center} 
  \caption{\label{fig:pinSL}
    The pinning force of solid-liquid interface v.s. the difference of the work of adhesion of two area:lyophilic and lyophobic part
,this relation is shown as Eq.~\eqref{eq:pin1}.
  }
\end{figure}

\section{conclusion}
We carried out MD simulations of a quasi-2d Lennard-Jones liquid 
between two parallel flat walls both of which had wetting discontinuous lines.
We calculated the pinning force and the contact angle% of a system with contact lines of three phases (solid,liquid,and vapor)
to show qualitatively
 the validity of the revised Young's equation which we derived theoretically
by accounting for the force balance on the control volume including the contact line.
The resultant showed that the pinning force at contact line had a saturation value and reached it
when the wetting discontinuous line was covered by liquid.
We made a system with only liquid phase and solid phase which had the wetting discontinuous lines
to examine the saturation value.
We concluded that the pinning force equals to the difference of the work of adhesion 
at two wetting part divided at the discontinuous line according to MD calculation and our old research data.

%flat smooth surface の先行研究に対して,flat でchemicalにinhomogeneousな面でのピニングを観察した.
%現実的な壁面はflatでない面だから,かなり理想的な場合にしか今回の研究は適応できないが,拡張をすることで
%より一般的なphysicalやchemicalなpinningを評価できるかもしれない.
%また固液界面が曲率を有したときなどは,smoothでも壁面原子が離散的に存在していること
%によってCNT等のナノ構造物を液体が濡れ広がるときにpinning現象が起きる可能性もある.%
%濡れ広がるような動的な接触角に固体壁面の状態が与える影響なども考えるべき.

%今回のは静止平衡,flat surface
%
%\section{Concluding remarks}
%
%% If you have acknowledgments, this puts in the proper section head.
%\begin{acknowledgments}
% Put your acknowledgments here.
%\end{acknowledgments}
%
%
\bibliography{./reference.bib}
%
%\end{figure}
%
\end{document}
%
% ****** End of file aiptemplate.tex ******

In order to examine pinning phenomenon induced by inhomogeneous wettability of wall surface as shown in Fig.~\ref{fig:systems}(a),
we employed two types of equilibrium simulation systems 
with respect to flat solid surfaces with inhomogeneous wettability, as shown in Fig.~\ref{fig:systems}:(b) for the contact line pinning and (c) liquid pinning calculation.
MD simulation method is almost same as the past paper~\cite{Yamaguchi2019} .
In all calculation systems, we used mono-atomic fluid molecules(Argon) and mono-atomic wall molecules 
whose structure is face-centered crystal(FCC).
Hereafter, fluid and wall molecules are denoted by `f' and `w', respectively.

\begin{figure}
  \begin{center}
    \includegraphics[width=100mm]{./figs/fig1.pdf}
  \end{center} 
  \caption{\label{fig:systems}
(a)Schematic of pinning induced by heterogeneous wettability.
    Lennard-Jones(LJ) liquid simulation systems on flat surfaces 
with inhomogeneous wettabilty which induces pinning of (b)the contact line
and (c)liquid.
  }
\end{figure}

\subsection{Potential}
Generic particles interacting through a 12-6 Lennard-Jones potential were adopted as the fluid molecules 
for ease of physical understanding as in our previous study.
The 12-6 LJ potential given by 
\begin{equation}
  \label{eq:LJ}
  \Phi^\mathrm{LJ}(r_{ij})=
  4\epsilon_\mathrm{ff} \left[
    \left(\frac{\sigma_\mathrm{ij}}{r_{ij}}\right)^{12} 
    -
    \left(\frac{\sigma_\mathrm{ij}}{r_{ij}}\right)^{6} 
   \right]
\end{equation}
was used as fluid-fluid and fluid-wall interaction.
In this 12-6 LJ potential, $r_{ij}$ was the distance between the
molecules $i$ at position 
$\bm{r}_{i}$ and $j$ at $\bm{r}_{j}$, while 
$\epsilon_\mathrm{ij}$ and $\sigma_\mathrm{ij}$ 
denoted the LJ energy and length parameters, 
respectively. 
This LJ interaction was truncated at a cut-off 
distance of $r^\mathrm{c}_\mathrm{ij}=3.5\,\sigma_\mathrm{ij}$ 
and quadratic functions were added so that the potential 
and interaction force smoothly vanished at $r_{c}$
as same as in our previous study~\cite{Yamaguchi2019}.

As above, the structure of solid wall is FCC and
 its inter-atomic potential is expressed by the following harmonic potential given by
\begin{equation}
\Phi^\mathrm{harmonic}(r_{ij})=
\frac{k}{2}(r_{ij}-r_{0})^{2},
\end{equation}
where $k$ and $r_{0}$ denote the spring constant and equilibrium distance, respectively.
These parameters and the mass of the wall atom corresponding with the property of platinum.

In addition to these intermolecular potentials, we used a one-dimentional field given by
\begin{equation}
\label{eq:Potwall}
\Phi^\mathrm{P}(z'_{i})=
4\pi\rho_{\text{n}}\epsilon^{0}_{fw}\sigma_{fw}^{2}\left[
\frac{1}{5}\left( \frac{\sigma_{fw}}{z'_{i}}\right)^{10}
-\frac{1}{2}\left( \frac{\sigma_{fw}}{z'_{i}}\right)^{4}
\right]
\end{equation}
as a function of the distance $z'_{i}=|z_{i}-z_{\text{pw}}|$
between a fluid particle and a potential wall.
The potential and mass parameters are summarized in Table~\ref{tab:table1}.

\subsection{Simulation systems}
For both of the calculation systems, liquid bath was put between two solid walls parallel to yz-plane and
periodic boundary conditions were set in y-direction as shown in Fig.~\ref{fig:systems}(b)(c),
so that liquid was shaped quasi two dimension: the effect of line tension was neglected.
The position of the outermost layers of solid walls consisting FCC was fixed, 
and the temperature of the second outermost layers was controlled 
by using the standard Langevin thermostat in all directions at a control temperature $T_{\text{c}}$
with a Debye temperature $T_{\text{Debye}}$.
Solid walls have eight layers 
so that solid wall atoms of  the second outermost layer don't interact with liquid molecules.

The wettability of the solid walls is inhomogeneous, lower and upper layers are lyophilic and lyophobic, respectively.
Solid liquid interaction coefficients of lyophilic and lyophobic parts are $\eta_{\text{phil}}$ and $\eta_{\text{phob}}$, respectively.
For contact line pinning system, a mirror boundary condition was set at the top boundary in z-direction
and a potential wall with a potential field Eq.~\eqref{eq:Potwall} was fixed at various positions $z_{\text{pw}}$,
so that we made equillibrium systems in Fig.~\ref{fig:systems}(a).
For liquid pinning system, the potential wall was fixed at a position $z_{\text{pw}}=1.4$~nm,
and additionally we used a piston with a potential field Eq.~\eqref{eq:Potwall} to make liquid single-phase 
under the following equation of motion:
\begin{equation}
m_{\text{pis}}\frac{d^{2} z_{\text{pis}}}{dt^{2}}=F_{\text{pis}}^{\text{int}}-A_{\text{pis}}p_{\text{pis}}
\end{equation}
where $z_{\text{pis}}$, $A_{\text{pis}}$, $m_{\text{pis}}$ and $p_{\text{set}}$ 
is a position, area, weight and control pressure of piston,respectively.
In the liquid pinning system, the piston cannot control the liquid pressure exactly
because the piston contacts with the wall surface inducing the negligible effect of the solid liquid interfacial tension.
The internal force on the piston $F_{\text{pis}}^{\text{int}}$ is the total normal force exerted by fluid molecules.
%with a potential field Eq.~\eqref{eq:Potential} at $z_{\text{pw}}=z_{\text{pis}}$

The velocity Verlet method was applied for the integration of the Newtonian equation of motion 
with a time increment $\Delta t$ of 5 fs for all systems. 
Values of the simulation parameters are summarized in Table~\ref{tab:table1} with the corresponding non- dimensional ones, 
which are normalized by the corresponding standard values.  

\begin{table}[!t]
\caption{\label{tab:table1} 
Simulation parameters. Corresponding non-dimensional values
are shown below.
}
%\begin{tabular*}{100mm}{@{\extracolsep{\fill}}|c|c|c|} \hline\hline
\begin{ruledtabular}
\begin{tabular}{cccc}
property  & value & unit & non-dim. value
\\ \hline
$\sigma_\mathrm{ff}$ & 0.340 & nm & 1
\\
$\sigma_\mathrm{ww}$ 
\footnote{Used only for Lorentz mixing rule.}
& 0.350 & nm & 1.03 
\\
$\sigma_\mathrm{fw}$ & $(\sigma_\mathrm{ff} + \sigma_\mathrm{ww})/2$ 

\\
$\epsilon_\mathrm{ff}$ & $1.67 \times 10^{-21}$ & J & 1
\\
$\epsilon_\mathrm{ww}$ 
\footnote{Used only for Berthelot mixing rule.}
& $1.000\times 10^{-21}$ & J & 0.599
\\
$\epsilon^{0}_\mathrm{fw}$ & 
$ \sqrt{\epsilon_\mathrm{ff}\epsilon_\mathrm{ww}}$
\\
$\epsilon_\mathrm{fw}$ & 
$ \eta_{\text{phil}} \epsilon^{0}_\mathrm{fw}$ or $\eta_{\text{phob}} \epsilon^{0}_\mathrm{fw}$
\\
$\eta_{\text{phil}}$ &
0.5& - & -
\\
$\eta_{\text{phob}}$ &
0.001 - 0.5& - & -
\\
$m_\mathrm{f}$ & $6.63 \times 10^{-26}$ & kg & 1
\\
$m_\mathrm{w}$ & $32.4\times 10^{-26}$ & kg & 4.88
\\
$m_{\text{pis}}$ & $14.7\times 10^{-22}$ & kg & $2.22\times10^{4}$
\\
$r_0$ (FCC) & $0.277$ & nm & 0.815
\\
$k$ & 46.8  & N/m & 3.24$\times10^{3}$
\\
$T_\mathrm{c}$ & 85  & K & 0.703
\\
$N_\mathrm{f}$ (contact line pinning) & 5000  & - & -
\\
$N_\mathrm{f}$ (liquid pinning) & 10000 & - & -
\\
$p_\mathrm{set}$ &1.00$\times 10^{6}$ & nm & 2.35$\times10^{-2}$
\\
$A_\mathrm{pis}$ &43.6 & $\text{(nm)}^{2}$ & 3.77$\times10^{2}$
\end{tabular}
\end{ruledtabular}
%
\end{table}
